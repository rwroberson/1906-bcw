\P\textit{Baptism is not to be unnecessarily delayed; nor to be administered in any case by any private perosn; but by a Minister of Christ, called to be a steward of the mysteries of God.}--Directory for Worship, Chap. viii, \S 1. \\ 

\noindent\P\textit{Although Baptism is usually to be administered in the church, in the presence of the congregation; yet there may be cases when it will be expedient to administer this Sacrament in private houses; of which the Minister is to be the judge.}--Directory for Worship, Chap. viii, \S 5. \\ 

\noindent\P\textit{After previous notice is given to the Minister, the Child to be baptized is to be presented to him by one or both the parents, whom the Minister shall address in this wise:} \\ 

\lettrine{D}{early} beloved, Almighty God, who has called us into His Church, has promised to be our God, and also the God and Father of our children; which covenant He renews in this Sacrament of Baptism, given to us and to our children, as a sign and seal of the washing away of our sins and our ingrafting into Christ.
St. Paul assures us that the children of the faithful are to be numbered among the holy people of God.
Our Saviour also, in the Gospel, calls the children unto Him, and blessed them, saying: Suffer the little children to come unto Me, and forbid them not: for of such is the Kingdom of God.

Forasmuch as you desire and claim these blessings for your \textit{Child,} you will now engage, on your part, to perform those things which God requires of you, that the good will and pleasure of your Heavenly Father may not be hidden from your \textit{Child.} \\  

\noindent\P\textit{Here the Minister shall address the following} \textsc{Questions} \textit{to the Parents; and the Parents, each of them, shall make answer:} \\

\begin{tabular}{l l}
	\textit{Question.} & Do you accept, for yourself and for your \\
					& \textit{Child,} the covenant of God, and therein \\
					& consecrate your \textit{Child} to Him? \\
	\textit{Answer.} & I do. \\
	\textit{Question.} & Do you promise to instruct your \textit{Child} in \\
					& the principles of our holy religion, as contained \\
					& in the Scriptures, to pray with \textit{him} and for \textit{him,} \\
					& and to bring \textit{him} up in the nurture and \\
					& admonition of the Lord? \\    
	\textit{Answer.} & I do. \\
\end{tabular}
\vspace{1ex}

{\centering \P\textit{Then the Minister shall say,} \par} 
\vspace{1ex}

\lettrine{G}{rant,} O Lord, to \textit{these} Thy \textit{servants} grace to perform the things which \textit{they have} promised before Thee:
And sactify with thy Spirit \textit{this Child} now to be baptized according to Thy word; through Jesus Christ our Lord.
\textit{Amen.} \\ 

\noindent\P\textit{Then, all present reverently standing, the Minister shall say to the Parents,}

{\centering What is the name of this Child? \par}
\vspace{1ex}

\noindent\P\textit{Then the Minister (taking the Child in his arms, or leaving it in the arms of the Parent), pronouncing the name of the Child, shall pour or sprinkle water upon it, saying,} \\

\lettrine{N}{---}, I baptize thee in the Name of the Father, and of the Son, and of the Holy Ghost.
\textit{Amen.} \\ 

\begin{center}
\P\textit{Then the Minister shall say,} \\
\vspace{1ex}
Let us pray. \\
\end{center}

\lettrine{M}{ost} holy and merciful Father, we give Thee hearty thanks that Thou hast numbered us amongst Thy people, and dost call our children unto Thee, marking them with this Sacrament, as a singular token and badge of Thy love.
Wherefore, we beseech Thee to confirm Thy favour more and more toward us, and to take into Thy tuition and defense \textit{this Child,} whom we offer and present unto Thee with common supplications.
Grant that \textit{he} may know Thee \textit{his} merciful Father, through Thy Holy Spirit working in \textit{his heart,} and that \textit{he} may not be ashamed to confess the faith of Christ crucified; but may continue His faithful soldier and servant, and so prevail against evil that in the end \textit{he} may obtain the victory, and be exalted into the liberty of Thy kingdom; through Jesus Christ our Lord.
\textit{Amen.} \\ 

\noindent\P\textit{Then the Minister and People may say together the} \textsc{Lord's Prayer,} \textit{if the same is not said in the Service immediately preceding or following.}   

\section*{The Lord's Prayer}

\lettrine{O}{ur} Father which art in heaven, Hallowed be Thy Name.
Thy kingdom come.
Thy will be done in earth, As it is in heaven.
Give us this day our daily bread.
And forgive us our debts, As we forgive our debtors.
And lead us not into temptation, But deliver us from evil:
For Thine is the kingdom, and the power, and the glory, for ever.
Amen.

{\centering \P\textit{Then the Minister shall say,} \par} 
\vspace{1ex}

\lettrine{T}{he} grace of the Lord Jesus Christ, and the love of God, and the communion of the Holy Ghost, be with you all.
\textit{Amen.} \\ 

\noindent\P\textit{Infants descending from parents, either both or but one of them professing faith in Christ and obedience to Him, are within the covenant of promise, and are to be baptized.}--Larger Catechism, 166. \\ 

\noindent\P\textit{The efficacy of Baptism is not tied to that moment of time wherein it is administered; yet, notwithstanding, by the right use of this Sacrament the grace promised is not only offered, but really exhibited and conferred by the Holy Ghost, to such (whether of age or infants) as that grace belongeth unto, according to the counsel of God's own will, in His appointed time.
Grace and salvation are, however, not so inseparably annexed unto Baptism as that none can be regenerated or saved without it, or that all that are baptized are undoubtedly regenerated.}--Confession of Faith, Chap. xxviii, \S\S 5, 6. \\ 

\noindent\P\textit{When, by death of the parents or otherwise, children are removed from their custody, the guardian or other person who has undertaken to rear them may present them for Baptism, provided} he \textit{posses the qualifications requesite for having} his \textit{own children baptized, and is willing to assume the obligations made by parents in the foregoing service.}--Minutes of the Synod of 1786.   
