\P \textit{None of the Forms of Service in this Book are intended to be in any sense obligatory; but where a given Order is voluntarily used it will promote unity and the general convenience if the parts are followed as here arranged.} \\

\noindent\P \textit{In order that all the People should take their part in the Public Worship of God, it is most earnestly recommended that, in those churches which choose to make use of the following Orders and Forms of Service, every member of the Congregation should be supplied with a copy of thie Book; and, also, that the directions which precede each part of the different services should be studied beforehand both by the Minister and People, so that confusion or uncertainty in the conduct of Worship may be avoided.} \\

\noindent\P \textit{The playing of a voluntary on the organ while the Congregation are assembling, the singing of a Psalm instead of the responsive reading from the} Psalter, \textit{or of an} Anthem \textit{after the Scripture reading, and other like observances, are left to the choice of each church.
But it chould always be rememberd that the Organist and the Choir are members of the Congregation, and that their cooperation will be of the greatest help to the Minister in conducting all the parts of Worship.} \\

\noindent\P \textit{To attain the prompt and hearty participation of the People, it is necessary that they should know beforehand what is expected of them.
Therefore, when any change is to be made in the order of a Service as given in the Book (as, for example, in the use of} The Commandments \textit{at the Morning Service,} The Beatitudes \textit{at the Evening Service, or in the time of making the} Offering \textit{or of singing the last} Hymn\textit{), the Minister should be careful that the Congregation are duly notified.
It is also desirable, in places where it is convenient, that the numbers of the Hymns and of the Selection from the Psalter be posted where they can be seen by the Congregation.} \\

\noindent\P \textit{In connection with the} General Prayer \textit{in the Lord's Day Services it is recommended that the Minister make faithful preparation of his own heart and mind to lead the People in their} Adorations, Thanksgivings, Supplications \textit{and} Intercessions.
\textit{If he should make any use, in thus preparing himself, of those which are given in the} \textsc{Treasury of Prayers,} \textit{this should be done with forethought and much care, in order not only to avoid injudicious length, but also that the prayer may be framed to express his own thought and feeling, and may also be suited to the occasion or the special need of the congregation.} \\

\noindent\P \textit{The use of this Book will be profitable only to those who are careful not to make it a means of formal or restricted worship, either public or private; but who remember always their duty as Christians both to seek and to cherish the gift of Prayer, by which they shall be enabled to frame wise and earnest and reverent petitions, as well for others as for themselves.
And in very service of the Church it is fitting that the Minister and every one of the People should pray in silence, at the beginning, for the guidance and help of the Holy Spirit, and at the close of the service, that all who have taken part in it may receive the blessing of God through Jesus Christ.} \\
